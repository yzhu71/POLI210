% Options for packages loaded elsewhere
\PassOptionsToPackage{unicode}{hyperref}
\PassOptionsToPackage{hyphens}{url}
%
\documentclass[
]{article}
\usepackage{amsmath,amssymb}
\usepackage{iftex}
\ifPDFTeX
  \usepackage[T1]{fontenc}
  \usepackage[utf8]{inputenc}
  \usepackage{textcomp} % provide euro and other symbols
\else % if luatex or xetex
  \usepackage{unicode-math} % this also loads fontspec
  \defaultfontfeatures{Scale=MatchLowercase}
  \defaultfontfeatures[\rmfamily]{Ligatures=TeX,Scale=1}
\fi
\usepackage{lmodern}
\ifPDFTeX\else
  % xetex/luatex font selection
\fi
% Use upquote if available, for straight quotes in verbatim environments
\IfFileExists{upquote.sty}{\usepackage{upquote}}{}
\IfFileExists{microtype.sty}{% use microtype if available
  \usepackage[]{microtype}
  \UseMicrotypeSet[protrusion]{basicmath} % disable protrusion for tt fonts
}{}
\makeatletter
\@ifundefined{KOMAClassName}{% if non-KOMA class
  \IfFileExists{parskip.sty}{%
    \usepackage{parskip}
  }{% else
    \setlength{\parindent}{0pt}
    \setlength{\parskip}{6pt plus 2pt minus 1pt}}
}{% if KOMA class
  \KOMAoptions{parskip=half}}
\makeatother
\usepackage{xcolor}
\usepackage[margin=1in]{geometry}
\usepackage{color}
\usepackage{fancyvrb}
\newcommand{\VerbBar}{|}
\newcommand{\VERB}{\Verb[commandchars=\\\{\}]}
\DefineVerbatimEnvironment{Highlighting}{Verbatim}{commandchars=\\\{\}}
% Add ',fontsize=\small' for more characters per line
\usepackage{framed}
\definecolor{shadecolor}{RGB}{248,248,248}
\newenvironment{Shaded}{\begin{snugshade}}{\end{snugshade}}
\newcommand{\AlertTok}[1]{\textcolor[rgb]{0.94,0.16,0.16}{#1}}
\newcommand{\AnnotationTok}[1]{\textcolor[rgb]{0.56,0.35,0.01}{\textbf{\textit{#1}}}}
\newcommand{\AttributeTok}[1]{\textcolor[rgb]{0.13,0.29,0.53}{#1}}
\newcommand{\BaseNTok}[1]{\textcolor[rgb]{0.00,0.00,0.81}{#1}}
\newcommand{\BuiltInTok}[1]{#1}
\newcommand{\CharTok}[1]{\textcolor[rgb]{0.31,0.60,0.02}{#1}}
\newcommand{\CommentTok}[1]{\textcolor[rgb]{0.56,0.35,0.01}{\textit{#1}}}
\newcommand{\CommentVarTok}[1]{\textcolor[rgb]{0.56,0.35,0.01}{\textbf{\textit{#1}}}}
\newcommand{\ConstantTok}[1]{\textcolor[rgb]{0.56,0.35,0.01}{#1}}
\newcommand{\ControlFlowTok}[1]{\textcolor[rgb]{0.13,0.29,0.53}{\textbf{#1}}}
\newcommand{\DataTypeTok}[1]{\textcolor[rgb]{0.13,0.29,0.53}{#1}}
\newcommand{\DecValTok}[1]{\textcolor[rgb]{0.00,0.00,0.81}{#1}}
\newcommand{\DocumentationTok}[1]{\textcolor[rgb]{0.56,0.35,0.01}{\textbf{\textit{#1}}}}
\newcommand{\ErrorTok}[1]{\textcolor[rgb]{0.64,0.00,0.00}{\textbf{#1}}}
\newcommand{\ExtensionTok}[1]{#1}
\newcommand{\FloatTok}[1]{\textcolor[rgb]{0.00,0.00,0.81}{#1}}
\newcommand{\FunctionTok}[1]{\textcolor[rgb]{0.13,0.29,0.53}{\textbf{#1}}}
\newcommand{\ImportTok}[1]{#1}
\newcommand{\InformationTok}[1]{\textcolor[rgb]{0.56,0.35,0.01}{\textbf{\textit{#1}}}}
\newcommand{\KeywordTok}[1]{\textcolor[rgb]{0.13,0.29,0.53}{\textbf{#1}}}
\newcommand{\NormalTok}[1]{#1}
\newcommand{\OperatorTok}[1]{\textcolor[rgb]{0.81,0.36,0.00}{\textbf{#1}}}
\newcommand{\OtherTok}[1]{\textcolor[rgb]{0.56,0.35,0.01}{#1}}
\newcommand{\PreprocessorTok}[1]{\textcolor[rgb]{0.56,0.35,0.01}{\textit{#1}}}
\newcommand{\RegionMarkerTok}[1]{#1}
\newcommand{\SpecialCharTok}[1]{\textcolor[rgb]{0.81,0.36,0.00}{\textbf{#1}}}
\newcommand{\SpecialStringTok}[1]{\textcolor[rgb]{0.31,0.60,0.02}{#1}}
\newcommand{\StringTok}[1]{\textcolor[rgb]{0.31,0.60,0.02}{#1}}
\newcommand{\VariableTok}[1]{\textcolor[rgb]{0.00,0.00,0.00}{#1}}
\newcommand{\VerbatimStringTok}[1]{\textcolor[rgb]{0.31,0.60,0.02}{#1}}
\newcommand{\WarningTok}[1]{\textcolor[rgb]{0.56,0.35,0.01}{\textbf{\textit{#1}}}}
\usepackage{graphicx}
\makeatletter
\def\maxwidth{\ifdim\Gin@nat@width>\linewidth\linewidth\else\Gin@nat@width\fi}
\def\maxheight{\ifdim\Gin@nat@height>\textheight\textheight\else\Gin@nat@height\fi}
\makeatother
% Scale images if necessary, so that they will not overflow the page
% margins by default, and it is still possible to overwrite the defaults
% using explicit options in \includegraphics[width, height, ...]{}
\setkeys{Gin}{width=\maxwidth,height=\maxheight,keepaspectratio}
% Set default figure placement to htbp
\makeatletter
\def\fps@figure{htbp}
\makeatother
\setlength{\emergencystretch}{3em} % prevent overfull lines
\providecommand{\tightlist}{%
  \setlength{\itemsep}{0pt}\setlength{\parskip}{0pt}}
\setcounter{secnumdepth}{-\maxdimen} % remove section numbering
\ifLuaTeX
  \usepackage{selnolig}  % disable illegal ligatures
\fi
\IfFileExists{bookmark.sty}{\usepackage{bookmark}}{\usepackage{hyperref}}
\IfFileExists{xurl.sty}{\usepackage{xurl}}{} % add URL line breaks if available
\urlstyle{same}
\hypersetup{
  pdftitle={Probability Notebook},
  pdfauthor={Yuhang Zhu},
  hidelinks,
  pdfcreator={LaTeX via pandoc}}

\title{Probability Notebook}
\author{Yuhang Zhu}
\date{2023-10-08}

\begin{document}
\maketitle

\hypertarget{chapter-1-probability-and-counting}{%
\subsubsection{Chapter 1 Probability and
Counting}\label{chapter-1-probability-and-counting}}

\begin{enumerate}
\def\labelenumi{\arabic{enumi}.}
\tightlist
\item
  De Morgan's laws:
\end{enumerate}

\begin{itemize}
\tightlist
\item
  \((A \cup B)^c = A^c \cap B^c\)
\item
  \((A \cap B)^c = A^c \cup B^c\)
\end{itemize}

\begin{enumerate}
\def\labelenumi{\arabic{enumi}.}
\setcounter{enumi}{1}
\tightlist
\item
  Multiplication rule: \(a*b\)
\end{enumerate}

\begin{itemize}
\tightlist
\item
  With replacement: \(n^k\)
\item
  Without replacement: \(n(n - 1)...(n-k+1)\)
\item
  A set with n elements has \(2^n\) subsets, including the empty set;
  and the set itself.
\end{itemize}

\begin{enumerate}
\def\labelenumi{\arabic{enumi}.}
\setcounter{enumi}{2}
\tightlist
\item
  Adjusting for overcounting
\end{enumerate}

\begin{itemize}
\tightlist
\item
  Combination (Binomial coefficient): For any nonnegative integers
  \emph{k} and \emph{n}, the \emph{binomial coefficient}
  \({n \choose k}\), is the number of subsets of size \emph{k} for a set
  of size \emph{n}.
  \[ C^{n}_{k} = {n \choose k} = \frac{n(n - 1) \ldots (n-k+1)}{k!} = \frac{n!}{(n - k)!k!} \]
\item
  Binomial theorem:
  \[ (x + y)^n = \sum_{k=0}^{n}{n \choose k}x^ky^{n-k} \]
\end{itemize}

\begin{enumerate}
\def\labelenumi{\arabic{enumi}.}
\setcounter{enumi}{3}
\tightlist
\item
  Equations
\end{enumerate}

\begin{itemize}
\item
  choosing the complement \[ {n \choose k} = {n \choose n-k} \]
\item
  team captain \[ n{n-1 \choose k-1} = k{n \choose k} \]
\item
  Vandermonde's identity
  \[ {m+n \choose k} = \sum_{j=0}^{k}{m \choose j}{n \choose k-j} \]
\item
  partnerships
  \[ \frac{(2n)!}{2^n \cdot n!} = (2n-1)(2n-3) \ldots 3 \cdot 1 \]
\end{itemize}

\begin{enumerate}
\def\labelenumi{\arabic{enumi}.}
\setcounter{enumi}{4}
\tightlist
\item
  Properties of probability
\end{enumerate}

\begin{itemize}
\tightlist
\item
  \(P(A^c) = 1 - P(A)\)
\item
  if \(A \subseteq B\), then \(P(A) \leq P(B)\)
\item
  \(P(A \cup B) = P(A) + P(B) - P(A \cap B)\)
\end{itemize}

\begin{enumerate}
\def\labelenumi{\arabic{enumi}.}
\setcounter{enumi}{5}
\tightlist
\item
  de Montmort's matching problem
\end{enumerate}

\begin{Shaded}
\begin{Highlighting}[]
\FunctionTok{set.seed}\NormalTok{(}\DecValTok{123}\NormalTok{)  }\CommentTok{\# 设置随机数生成种子以获得可重复的结果}
\NormalTok{sims }\OtherTok{\textless{}{-}} \DecValTok{100000}  \CommentTok{\# 设置模拟次数}
\NormalTok{num\_cards }\OtherTok{\textless{}{-}} \DecValTok{52}  \CommentTok{\# 设置牌的数量}

\CommentTok{\# 初始化一个变量来存储每次模拟的结果}
\NormalTok{matches }\OtherTok{\textless{}{-}} \FunctionTok{numeric}\NormalTok{(sims)}

\CommentTok{\# 进行模拟}
\ControlFlowTok{for}\NormalTok{ (i }\ControlFlowTok{in} \DecValTok{1}\SpecialCharTok{:}\NormalTok{sims) \{}
\NormalTok{  original\_order }\OtherTok{\textless{}{-}} \DecValTok{1}\SpecialCharTok{:}\NormalTok{num\_cards  }\CommentTok{\# 创建一个表示原始顺序的向量}
\NormalTok{  shuffled\_order }\OtherTok{\textless{}{-}} \FunctionTok{sample}\NormalTok{(original\_order)  }\CommentTok{\# 随机洗牌}
  \CommentTok{\# 检查是否有任何一张牌在其原始位置上,并存储结果}
\NormalTok{  matches[i] }\OtherTok{\textless{}{-}} \FunctionTok{any}\NormalTok{(original\_order }\SpecialCharTok{==}\NormalTok{ shuffled\_order)}
\NormalTok{\}}

\CommentTok{\# 计算至少有一张牌在其原始位置上的概率}
\FunctionTok{mean}\NormalTok{(matches)}
\end{Highlighting}
\end{Shaded}

\begin{verbatim}
## [1] 0.63436
\end{verbatim}

\hypertarget{chapter-2-conditional-probability}{%
\subsubsection{Chapter 2 Conditional
Probability}\label{chapter-2-conditional-probability}}

\begin{enumerate}
\def\labelenumi{\arabic{enumi}.}
\item
  Definition of Conditional probability\\
  If A and B are events with P(B) \textgreater{} 0, then the conditional
  probability of A given B, denoted by P(A\textbar B), is defined as:
  \[ P(A|B) = \frac{P(A \cap B)}{P(B)} \]
\item
  For any events A and B with positive probabilities,
  \[ P(A \cap B) = P(B)P(A|B) = P(A)P(B|A) \]
\item
  Bayes' rule \[ P(A|B) = \frac{P(B|A)P(A)}{P(B)} \]
\end{enumerate}

\end{document}
